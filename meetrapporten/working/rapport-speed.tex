\documentclass[a4paper]{article}
\usepackage{amsmath}
\usepackage{hyperref}

\usepackage{listings}
\usepackage{color}

\definecolor{dkgreen}{rgb}{0,0.6,0}
\definecolor{gray}{rgb}{0.5,0.5,0.5}
\definecolor{mauve}{rgb}{0.58,0,0.82}

\lstset{frame=tb,
  language=Java,
  aboveskip=3mm,
  belowskip=3mm,
  showstringspaces=false,
  columns=flexible,
  basicstyle={\small\ttfamily},
  numbers=none,
  numberstyle=\tiny\color{gray},
  keywordstyle=\color{blue},
  commentstyle=\color{dkgreen},
  stringstyle=\color{mauve},
  breaklines=true,
  breakatwhitespace=true,
  tabsize=3
}

\begin{document}
\title{*Meetrapport* speed across multiple methods of grey-scaling}
\author{Niels de Waal (1698041), Jasper Smienk(1700502)}
\maketitle
\newpage

\tableofcontents
\newpage

\section{Target}
With this *meetrapport* we want to find out how fast each grey-scaling method is and compare them to each other and the default implementation.

These results can help improve the speed of the facial recognition because converting an image to grey-scale is one of the steps that need to be done.

\section{Hypothesis}
We suspect that of our methods the \textit{decomposition} will be the fastest, as it required very little computation. Followed by \textit{averaging} and lastly \textit{luma}.

We can't say anything about how they will perform against the default implementation, as we don't know how it works.

\section{Method}
For each grey-scaling method, we will run it 1000 times and see how long it took from the start to the end. We will make sure the tests are run on the same laptop and keep an eye out on the temperature to make sure it does not thermal-throttle.


\section{Results}

\section{*Verwerking*}

\section{Conclusion}

\section{Evaluation}

\begin{thebibliography}{9}
\end{thebibliography}
\end{document}
